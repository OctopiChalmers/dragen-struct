\section{Predicting Distributions} \label{sec:synthesis}

Characterizing the distribution of values of an arbitrary random generator is a
hard task.
%
It requires modeling every random choice that a generator could possibly make to
generate a value.
%
In a recent work \cite{DBLP:conf/haskell/MistaRH18}, we have shown that it is
possible to \emph{analytically} predict the average distribution of data
constructors produced by random generators automatically derived considering
only ADT definitions---like the one presented on Section
\ref{sec:randomtesting}.
%
For this purpose, we found that random generation of ADT values can be
characterized using the theory of \emph{branching processes} \cite{gw1875}.
%
This probabilistic theory was originally conceived to predict the growth and
extinction of royal family trees the Victorian Era, later being applied to a
wide variety of research areas.
%
In this work, we adapt this model to predict the average distribution of values
of random generators derived considering structural information coming from
functions' pattern matchings and abstract interfaces.

Essentially, a branching process is a special kind of Markov process that models
the evolution of a population of \emph{individuals of different kinds} across
discrete time steps known as \emph{generations}.
%
Each kind of individual is expected to produce an average number of offspring of
(possibly) different kinds from one generation to the next one.
%
Mista el at. \cite{DBLP:conf/haskell/MistaRH18} show that branching processes
can be adapted to predict the generation of ADT values by simply considering
each data constructor as a kind of its own.
%
In fact, any ADT value can be seen as a tree where each node represents a root
data constructor and has its sub-expressions as sub-trees---hence note the
similarity with family trees.
%
In this light, each tree level of a random value can be seen as a generation of
individuals in this model.


We characterize the numbers of constructors that a random generator produces in
the $n$-th generation as a vector $G_n$, a vector that groups the number of
constructors of each kind produced in that generation---in our |Html| example,
this vector has four components, i.e., one for each constructor.
%
%We note $G_n$ to the vector of real numbers that represents the distribution of
%
From branching processes theory, the following equation captures the expected
distribution of constructors at the generation $n$, noted $E[G_n]$, as follows:
%
\begin{align}
  E[G_n]^T = E[G_0]^T \cdot M^n
  \label{eqn:prediction}
\end{align}
%
Vector $E[G_0]$ represents the initial distribution of constructors that our
generator produces, which simply consists of the generation probability of each
one.
%
The interesting aspect of the prediction mechanism is encoded in the matrix $M$,
known as a the \emph{mean matrix} of this stochastic process.
%
$M$ is a squared matrix with as rows and columns as different data constructors
involved in the generation process.
%
Each element $M_{i,j}$ of this matrix encodes the average number of data
constructors of kind $j$ that gets generated in a given generation, provided that we
generated a constructor of kind $i$ at the previous one.
%
In this sense, this matrix encodes the ``branching'' behavior of our random
generation from one generation to the next one.
%
Each element of the matrix can be automatically calculated by exploiting ADT
definitions, % \cite{DBLP:conf/haskell/MistaRH18}
%
% ---specially the recursive fields---
%
as well as the individual probability of generating each constructor.
%
For instance, the average number of |Text| data constructors that we will
generate provided that we generated a |Join| constructor on the previous
level results:
%
\begin{align*}
  M_{\mathrm{Join, Text}} = 2\cdot p_{\mathrm{Text}}
\end{align*}
%
where $2$ is the number of holes present when generating a partial ADT value
|Join| (i.e., |Join hole hole|) and $p_{\mathrm{Text}}$ is the probability of
individually generating the constructor |Text|.
%
This reasoning can be used to build the rest of the mean matrix analogously.

%\todo[inline]{Complete this!}

% \begin{align*}
%   E \left[ P_n \right]
%   = \beta_{Html}^T \cdot \left( \frac{I - (M_{Html})^{n+1}}{I - M_{Html}}\right)
% \end{align*}

\subsection{Extending predictions for structural information}

In this work, we show how to naturally fit structural information beyond ADT
definitions into the prediction mechanism of branching processes.
%
Our realization is that it suffices to consider \emph{each different pattern
  matching and function call as a kind of individual on its own}.
%
In that manner, we can extend our mean matrix $M$ adding a row and a column for
each different pattern matching and function call as shown in Figure
\ref{fig:meanmatrix}.
%
\begin{wrapfigure}{r}{0.26\textwidth}
  \vspace{-8pt}
  \newcommand{\ph}{\phantom{\square}}
  \newcommand{\el}{\square}
  \centering
  \resizebox{4.5cm}{!}{
    \begin{tikzpicture}
      [ baseline=-0.5ex
      % , transform canvas={scale=0.9}
      , every left delimiter/.style={xshift=.5em}
      , every right delimiter/.style={xshift=-.5em}
      ]
      \matrix
      [ matrix of math nodes
      , column sep=-1pt
      , row sep=-1pt
      , ampersand replacement=\&
      , left delimiter={[}
      , right delimiter={]}
      ] (m)
      {
        \ph \& \ph \& \ph \& \ph \& \ph \& \ph \& \ph \& \ph \& \ph \\ % Text
        \ph \& \ph \& \ph \& \ph \& \ph \& \ph \& \ph \& \ph \& \ph \\ % Sing
        \ph \& \ph \& \ph \& \ph \& \ph \& \ph \& \ph \& \ph \& \ph \\ % Tag
        \ph \& \ph \& \ph \& \ph \& \ph \& \ph \& \ph \& \ph \& \ph \\ % Join
        \ph \& \ph \& \ph \& \ph \& \ph \& \ph \& \ph \& \ph \& \ph \\ % simplify#1
        \ph \& \ph \& \ph \& \ph \& \ph \& \ph \& \el \& \ph \& \ph \\ % simplify#1
        \ph \& \ph \& \ph \& \ph \& \ph \& \ph \& \ph \& \ph \& \ph \\ % simplify#1
        \ph \& \ph \& \ph \& \ph \& \ph \& \ph \& \ph \& \ph \& \ph \\ % simplify#1
        \ph \& \ph \& \ph \& \ph \& \ph \& \ph \& \ph \& \ph \& \ph \\ % simplify#1
      };
      % Color squares
      \fill[red, opacity=0.15]
      (m-1-1.north west)
      -|| (m-1-3.north east)
      -|| (m-3-3.south east)
      -|| ([xshift=1pt]m-3-1.south west)
      -|| cycle;
      \fill[blue, opacity=0.15]
      ([yshift=-1pt]m-4-1.north west)
      -|| (m-1-3.north east)
      -|| ([yshift=-1pt]m-1-9.south east)
      -|| (m-9-9.south east)
      -|| ([xshift=-1pt, yshift=0pt]m-9-1.south east)
      -|| cycle;
      % Upper labels
      \node[above,text depth=1pt] at (m-1-1.north) {$ C_1$};
      \node[above,text depth=1pt] at (m-1-2.north) {$ \cdots$};
      \node[above,text depth=1pt] at (m-1-3.north) {$ C_i$};
      \node[above,text depth=1pt] at (m-1-4.north) {$ P_1$};
      \node[above,text depth=1pt] at (m-1-5.north) {$ \cdots$};
      \node[above,text depth=1pt] at (m-1-6.north) {$ P_j$};
      \node[above,text depth=1pt] at (m-1-7.north) {$ F_1$};
      \node[above,text depth=1pt] at (m-1-8.north) {$ \cdots$};
      \node[above,text depth=1pt] at (m-1-9.north) {$ F_k$};
      % Left labels
      \node[left,overlay] at (m-1-1.west) {$ C_1$};
      \node[left,overlay] at ([xshift=-4pt, yshift=2pt]m-2-1.west) {$ \vdots$};
      \node[left,overlay] at (m-3-1.west) {$ C_i$};
      \node[left,overlay] at (m-4-1.west) {$ P_1$};
      \node[left,overlay] at ([xshift=-4pt, yshift=2pt]m-5-1.west) {$ \vdots$};
      \node[left,overlay] at (m-6-1.west) {$ P_j$};
      \node[left,overlay] at (m-7-1.west) {$ F_1$};
      \node[left,overlay] at ([xshift=-4pt, yshift=2pt]m-8-1.west) {$ \vdots$};
      \node[left,overlay] at (m-9-1.west) {$ F_k$};
      % Dashed lines
      \draw[densely dashed, semithick] ([xshift=-2pt]m-3-1.south west) -- ([xshift=2pt]m-3-9.south east);
      \draw[densely dashed, semithick] ([xshift=-2pt]m-6-1.south west) -- ([xshift=2pt]m-6-9.south east);
      \draw[densely dashed, semithick] ([yshift=2pt]m-1-3.north east) -- ([yshift=-2pt]m-9-3.south east);
      \draw[densely dashed, semithick] ([yshift=2pt]m-1-6.north east) -- ([yshift=-2pt]m-9-6.south east);
      % \draw[dotted, semithick] (m-3-1.west) -- ([xshift=3pt]m-3-3.west);
      % \draw[dotted, semithick] (m-1-3.north) -- ([yshift=-3pt]m-3-3.north);
      \draw[dotted, thick] (m-6-1.west) -- ([xshift=4pt]m-6-7.west);
      \draw[dotted, thick] (m-1-7.north) -- ([yshift=-3pt]m-6-7.north);
    \end{tikzpicture}
}
  \caption{Mean matrix $M$ including pattern matching and function calls
    information.}
  \label{fig:meanmatrix}
  \vspace{-6pt}
\end{wrapfigure}
%
Symbol $C_1\! \cdots C_i$ denotes constructors, $P_1\! \cdots P_j$ pattern
matchings, and $F_1\! \cdots F_k$ function calls.
%
The light-red colored matrix is what we had before, whereas the light-blue
colored cells are new---we encourage readers to obtain a colored copy of this
work.
%
% To extend $M$ to consider the new possible constructions, we need to calculate
% $M_{i,j}$ for every new combination of ``father'' and ``son'' constructions $i$
% and $j$, respectively.
% %

The new cells are filled as before: we need to consider the amount of holes when
generating partial pattern matching values and function calls as well as their
individual probabilities.
%
% Fortunately, it is possible to follow the same pattern as we did on our original
% work, this time considering the amount of recursive pattern variables if it the
% case of a pattern matching value being a father individual, and the amount of
% recursive arguments if we deal with a abstract interface function call as a
% father individual.
%
For instance, if we consider $P_j$ as the second pattern of function |simplify|
and $F_1$ as function |div|, then the marked cell above has the value
$2\cdot p_{div}$, i.e., the amount of holes in the partially generated pattern
|(Join (Join (Text s) hole) hole)|, where |s| is some random string, times the
probability to generate a call to function |div|.
%
The rest of this matrix can be computed analogously.

% the average amount of function calls to |div| that we will produce
% provided that we generated a value matching the second pattern of |simplify| as a
% father results $2\cdot p_{div}$.
%
% Similarly, the average amount of values satisfying the first pattern matching of
% |simplify| provided that we generated a |Tag| constructor as a father results
% $1 \cdot p_{simplify\# 1}$.

As another contribution, we found that the whole prediction process can be
factored in terms of two vectors $\beta$ and $\mathcal{P}$, such that $\beta$
represents the number of holes in each partial ADT value that we generate,
whereas $\mathcal{P}$ simply represents the probability of generating that
partial ADT value.
% (this probability can be easily calculated based on the
% generation frequencies used in the random generator).
%
Then, the equation (\ref{eqn:prediction}) can be rewritten as:
%
\begin{align*}
  E[G_n]^T = \beta^T \cdot (\beta \cdot \mathcal{P}^T)^{n}
\end{align*}
%
% The vector $\beta$ can be obtained by analyzing the shape (namely holes) of each
% source of structural information that we consider, and joining them together.
%
For instance, $\beta$ and $\mathcal{P}$ for our generation specification of HTML
values are as shown in Figure \ref{fig:vectors}.
%
We note |simplify1| and |simplify2| to the patterns occurring in the first and
second clauses of |simplify|, respectively.
%(Dashed lines indicate the composition of the diffent sub-vectors
%that are calculated separately).

\begin{figure}[t]
  \resizebox{!}{2cm}{
\begin{equation*}
  \hspace{20pt}
  \beta = \qquad \qquad \
  \begin{tikzpicture}
    [ baseline=-0.65ex
    , every left delimiter/.style={xshift=.5em}
    , every right delimiter/.style={xshift=-.5em}
    ]
    \matrix
    [ matrix of math nodes
    , column sep=0ex
    , row sep=-1pt
    , ampersand replacement=\&
    , left delimiter={[}
    , right delimiter={]}
    ] (v)
    {
      0 \\ % Text
      0 \\ % Sing
      1 \\ % Tag
      2 \\ % Join
      0 \\ % simplify#1
      2 \\ % simplify#1
      0 \\ % hr
      1 \\ % div
      1 \\ % bold
    };
    \draw[densely dashed] ([xshift=-3pt]v-4-1.south west) -- ([xshift=3pt]v-4-1.south east);
    \draw[densely dashed] ([xshift=-3pt]v-6-1.south west) -- ([xshift=3pt]v-6-1.south east);
    \node[left,overlay] at ([xshift=-2pt]v-1-1.west) {$\scriptstyle \mathrm{Text}$};
    \node[left,overlay] at ([xshift=-2pt]v-2-1.west) {$\scriptstyle \mathrm{Single}$};
    \node[left,overlay] at ([xshift=-2pt]v-3-1.west) {$\scriptstyle \mathrm{Tag}$};
    \node[left,overlay] at ([xshift=-2pt]v-4-1.west) {$\scriptstyle \mathrm{Join}$};
    \node[left,overlay] at ([xshift=-2pt]v-5-1.west) {$\scriptstyle \mathrm{simplify}\#1$};
    \node[left,overlay] at ([xshift=-2pt]v-6-1.west) {$\scriptstyle \mathrm{simplify}\#2$};
    \node[left,overlay] at ([xshift=-2pt]v-7-1.west) {$\scriptstyle \mathrm{hr}$};
    \node[left,overlay] at ([xshift=-2pt]v-8-1.west) {$\scriptstyle \mathrm{div}$};
    \node[left,overlay] at ([xshift=-2pt]v-9-1.west) {$\scriptstyle \mathrm{bold}$};
  \end{tikzpicture}
  \hspace{30pt}
  \mathcal{P} =
  \begin{tikzpicture}
    [ baseline=-0.65ex
    , every left delimiter/.style={xshift=.5em}
    , every right delimiter/.style={xshift=-.5em}
    ]
    \matrix
    [ matrix of math nodes
    , column sep=0ex
    , row sep=-1pt
    , ampersand replacement=\&
    , left delimiter={[}
    , right delimiter={]}
    ] (v)
    {
      p_{\mathrm{Text}} \\
      p_{\mathrm{Single}} \\
      p_{\mathrm{Tag}}  \\
      p_{\mathrm{Join}} \\
      p_{\mathrm{simplify\#1}} \\
      p_{\mathrm{simplify\#2}} \\
      p_{\mathrm{hr}} \\
      p_{\mathrm{div}} \\
      p_{\mathrm{bold}} \\
    };
    \draw[densely dashed] ([xshift=-15pt]v-4-1.south west) -- ([xshift=15pt]v-4-1.south east);
    \draw[densely dashed] ([xshift=-3pt]v-6-1.south west) -- ([xshift=3pt]v-6-1.south east);
  \end{tikzpicture}
\end{equation*}
}
\vspace{-5pt}
\caption{Prediction vectors of our |Html| generation specification.}
\label{fig:vectors}
\vspace{-10pt}
\end{figure}

Note that by varying the shape of the vector $\mathcal{P}$ we can tune the
distribution of our random generator in a way that can be always characterized
and predicted.
%
\dragenp follows a similar approach as \dragen and uses an heuristic to tune the
generation probabilities of each source of structural information.
%
This is done by running a simulation-based optimization process at compile-time.
%
This process is parameterized by the desired distribution of values set by the
user.
%
In this manner, developers can specify, for instance, a uniform distribution of
data constructors, pattern matching values and function calls or, alternatively,
a distribution of values with some constructions appearing in a different
proportion as others, e.g., two times more functions calls to |div| than |Join|
constructors.

\subsection{Overall prediction}
%\paragraph{End-to-End Prediction}
%
% So far we have shown that our prection model is able to predict the distribution
% of data constructors, composite pattern values and interface function calls that
% our automatically derived random generators will produce on average.

It is possible to provide an overall prediction of the expected number of
constructors when restricting the generation process to only bare data
constructors and pattern matching values.
%
To achieve that, we should stop considering pattern matching values as atomic
constructions and start seeing them as compositions of several data
constructors.
%
In that manner, it is possible to obtain the expected \emph{total} number of
generated data constructors that our generators will produce---regardless if they
are generated on their own, or as part of a pattern matching value.
%
We note this number as $E^\downarrow\![\_]$ and, to calculate it, we only need
to add the expected number of bare constructors that are included within each
pattern matching.
%
For instance, we can calculate the total expected number of constructors |Text|
and |Join| that we will generate by simply expanding the expected number of
generated pattern matching values |simplify1| and |simplify2| into their
corresponding data constructors:
%
{\small
  \begin{alignat*}{4}
    &E^\downarrow\!\left[ |Text| \right]
    &= E\left[ |Text| \right]
      &+ 2 \cdot E\!\left[ |simplify1| \right]\
      &+\ 1 \cdot E\!\left[ |simplify2| \right] \\
    &E^\downarrow\!\left[ |Join| \right]
    &= E\left[ |Join| \right]
      &+ 1 \cdot E\!\left[ |simplify1| \right]\
      &+\ 2 \cdot E\!\left[ |simplify2| \right]
  \end{alignat*}
}
%
Observe that each time we generate a value satisfying the first pattern matching
of the function |simplify|, we add two |Text| and one |Join| data constructors
to our random value.
%
The case of the second pattern matching of |simplify| follows analogously.
%
Note that the overall prediction cannot be applied if we also generate random
values containing function calls, as we cannot predict the output of an
arbitrary function.


% Local Variables:
% TeX-master: "main.lhs.tex"
% TeX-command-default: "Make"
% End:
