\section{Related Work}
\label{sec:related}

\citeauthor{SwierstraDTC} \cite{SwierstraDTC} proposes the Data Types \`a la
Carte approach for building extensible data types and unified interpreters as a
solution for the expression problem coined by \citeauthor{wadler1998expression}
\cite{wadler1998expression}.
%
In this work, we take ideas from this approach and extend them to work in the
scope of random data generation, where other parameters come into play apart
from just combining construction, like generation frequency and terminal
constructions.


From the practical point of view, \citeauthor{KiriyamaOptimizingDTC} propose an
optimization mechanism for Data Types \`a Carte, where a concrete data type is
derived for each different variant of the compositional data types we use in our
codebase \cite{KiriyamaOptimizingDTC}.
%
This solution avoids much of the runtime overhead introduced when internally
pattern matching against sequencies of |InL| and |InR| data constructors.
%
However, we see this solution as not entirely compositional, since we still need
to rely on Template Haskell to derive the machinery for each specialized data
type.
%
In our particular setting, we found that our solution has a fairly acceptable
overhead when the representation types are balanced.
%
Hence, our aim in the future is to keep our fully extensible approach, but
automatically computing the best arrangement of constructions in order to reduce
the overhead as much as possible.


\emph{DRAGEN} is a random testing derivation tool that synthesizes random
generators for our target data types, tuning their generation frequencies using
a simulation-based optimization process parametrized by the distribution of
values desired by the user \cite{DBLP:conf/haskell/MistaRH18}.
%
This simulation is based on the theory of \emph{branching processes}, which
models the growth and extinction of populations across succesive generations.
%
In this setting, populations consist of randomly generated data constructors,
where generators correspond to each level of the generated values.
%
This tool has shown to improve the code coverage obtained over our codebase,
with respect to other fully-automated generators derivation mechanism.
%
Recently, \citeauthor{Mista2019GeneratingRS} extended this approach to generate
random values considering also the other sources of structural information
covered here, namely abstract interfaces and function pattern machings.
%
Using this extension, the authors show how the usage of extra information when
generating random values is extremely valuable, particularly in situations like
the ones described in Section \ref{sec:sources}, where the usual derivation
approaches fail to synthesize useful generators due to a lack of structural
information.
%
Our work is inspired by the same general idea. In constrast, in this paper we
tackle the representation problem, showing how these ideas can be effectively
implemented in Haskell using advanced type-level features.
