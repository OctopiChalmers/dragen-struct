\section{Background} \label{sec:randomtesting}

In this section, we briefly introduce the common approach for automatically
deriving random data generators for ADTs in QuickCheck by using the information
found at the type-level.
%type-driven approach, along
%with its main drawbacks.
%
% Haskell is a strongly typed programming language with a powerful type system.
% %
% It lets programmers encode a considerable amount of information about the
% structure of their systems using data types that can be checked at compilation
% time.
% %
% One of its key aspects is the support for Algebraic Data Types (ADTs).
% %
% Essentially, an ADTs is a composite type defined by combining other types in
% terms of \textbf{sums} (also known as \emph{variant types}) and
% \textbf{products} (or tuples) of other data types.
% %
To exemplify this, and for illustrative purposes, let us consider the following
ADT definition to encoding simple |Html| pages:
%
\begin{code}
data Html  =   Text    String
           pp  Single  String
           pp  Tag     String  Html
           pp  Join    Html    Html

\end{code}
%
The type |Html| allows to build pages via four possible constructions:
|Text|---which represents plain text values---, |Single| and |Tag|---which
represent singular and paired Html tags, respectively---, and |Join|---which
concats two Htmls pages one after another.
%
% In the previous definition, we declare |Html| as the sum of four possible
% constructions: |Text| represents plain text values. |Sing| and |Tag| represent
% singular and paired Html tags, and |Join| concats two expression one after
% another.
% %
% We only encode a very small subset of the actual Html specification for
% illustrative reasons.
% %
In Haskell, |Text|, |Single|, |Tag|, and |Join| are known as \emph{data
  constructors} (or constructors for short) and are used to distiguish which
variant of the ADT we are constructing.
%
Each data constructor is defined as a product of zero or more types known as
\emph{fields}.
%
For instance, |Text| has a field of type |String|, whereas |Join| two recusive
fields of type |Html|.
%
In general, we will say that a data constructor with no recursive fields is
\emph{terminal}, and \emph{non-terminal} or \emph{recursive} if it has at least
one field of such nature.
%
With this representation, the page ``|html_example|'' can be encoded as:
%
\begin{code}
Tag "html" (Join (Join
  (Text "hello") (Sing "hr")) (Text "bye"))
\end{code}
%
% Additionally, we can define a function |render| to serialize |Html| values as
% follows:
% %
% \begin{code}
% render :: Html -> String
% render (Text t) = t
% render (Sing t) = "<" ++ t ++ ">"
% render (Join x y) = render x ++ render y
% render (Tag t x)
%   = "<" ++ t ++ ">" ++ render x ++ "</" ++ t ++ ">"
% \end{code}

% In the previous definition, |render| is described using \emph{pattern matching}
% over each possible kind of value.
% %
% Using pattern matching we can define functions idiomatically by defining
% different function clauses for each input pattern we are interested on.
% %
% Patterns can be defined to match specific constructors, literal values or
% variable subexpressions (like |t|, |x| and |y| in the definition of |render|).
% %
% They can also be nested in order to match very specific patterns of values.


\subsection{Type-driven generation of random values}

In order to generate random value of types involving ADTs, most approaches
require users to provide a random data generator for each one of them.
%
This is a cumbersome and error prone task and usually \emph{follows closely the ADT
structure}.
%
For instance, consider the following definition of a \quickcheck random
generator for the type |Html|:
%
\begin{code}
genHtml = sized (\size ->
   if size == 0  then  frequency
                       [ (2,  Text    <$> genString)
                       , (1,  Single  <$> genString) ]
   else frequency
     [  (2,  Text    <$> genString)
     ,  (1,  Single  <$> genString)
     ,  (4,  Tag     <$> genString <*> smaller genHtml)
     ,  (3,  Join    <$> smaller genHtml <*> smaller genHtml) ])
\end{code} %$
%
We use the Haskell syntax |[]| and |(,)| for denoting lists and pairs of
elements, respecitvely, e.g., |[(1,2),(3,4)]| is a list of pairs of numbers.
%
The random generator |genHtml| is defined using \quickcheck's function |sized|
to paremetrize the generation process up to an external natural number known as
the \emph{generation size}.
%
This parameter is chosen by the user, and it is used to limit the maximum amount
of recursive calls that this random generator can perform and thus ensuring
the termination of the generation process.
%
When called with a positive generation size, this generator can pick to generate
among any |Html| data constructor of with an explicitly given frequency that can
be chosen by the user---in this example, 2, 1, 4 and 3 for |Text|, |Single|,
|Tag|, and |Join|, respectively.
%
When it picks to generate a |Text| or a |Single| data constructor, it also
generates a random |String| value using the standard QuickCheck generator
|genString|.% (\quickcheck provides standard random generators for most base types
% like |String|, |Int|, |Bool|, etc.)
\footnote{%
  The operators |<$>| and |<*>| are used in Haskell to combine values obtained
  from calling to a random generator and they are not particularly relevant for
  the point being made in this work.}
%$
On the other hand, when it picks to generate a |Join| constructor, it also
generates two independent random subexpression recursively, thus decreasing the
generation size by a unit on each recursive invocation (|smaller genHtml|).
%
The case of random generation of |Tag| constructors follows analogously.
%
This random process keeps calling itself recursively until the generation size
reaches zero, where the generator is constrained to pick only among terminal
data constructors, being |Text| and |Single| the only possible choices in our
particular case.

% Strictly decreasing the the generation size by one on each recursive call
% results in a useful property over the generated data: every generated value has
% at most |n| levels, where |n| is the generation size set by the user.
%
% This property enables us to model the generation process using the theory of
% branching processes introduced by \tocite, and extended in this work as
% described in Section \ref{sec:synthesis}.


The previous definition is rather mechanical, except perhaps for the chosen
generation frequencies.
%
% In this light, it easy to extend this procedure to any data type defined in an
% algebraic fashion.
% %
% Fortunately, there exists different meta-programming tools to avoid the user
% from having to mechanically write random generators over and over again for each
% user-defined ADT.
% %
% The simplest tool for such purpose is \megadeth \tocite.
% %
% Given a target data type, it synthesizes a random generator for it that behaves
% similarly to the one presented above, where the generation frequencies are
% defined to be uniform across constructors.
% %
% However, picking among different data constructors with uniform frequency can
% lead to a generation process biased towards generating (in average) very small
% values, regardless of the generation sized set by the user \tocite.
%
Fortunately, there exists a tool called \dragen
\cite{DBLP:conf/haskell/MistaRH18} conceived to mitigate this problem.
%
Instead of setting a uniform frequency (or probability) of data constructors,
{\dragen} uses the theory of branching processes \cite{gw1875} to modelize and
predict analytically the expected number of generated data constructors.
%
This prediction mechanism is used to feedback a simulation-based optimization
process that adjusts the generation frequency of each data constructor in order
to obtain a particular distribution of values that can be specified by the
user---thus providing a flexible testing environment while still being mostly
automated.

As many other tools \tocite, {\dragen} synthesizes random generators similar to
the one shown before, where the generation process is limited to pick \emph{a
  single data constructor at the time and then recursively generate each
  required subexpression independently}.
%
% type. the limitations arise quickly when we consider that the underlying
% generation model is essentially the same:
%
% Althogh both \megadeth and \dragen synthesize random generators that are
% theoretically capable to generate the whole space of values of the target data
% type. the limitations arise quickly when we consider that the underlying
% generation model is essentially the same: they pick a single data constructor
% and recusively generate each required subexpression independently.
%
In practice, this procedure is often too generic to generate random data with
enough structural complexity required for testing certain applications.
% to be used for testing purposes.


% Local Variables:
% TeX-master: "main.lhs.tex"
% TeX-command-default: "Make"
% End:
