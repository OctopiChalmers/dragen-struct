\section{Introduction}

% Scenario of random testing
Random testing is a promising approach to finding bugs
\cite{HughesNSA16,HughesPAN16,ArtsHNS15}.
%
QuickCheck \cite{ClaessenH00} is a dominant tool of this sort used by the
Haskell community.
%
It requires developers to specify testing properties which describe programs
behavior and random data generator based on the \emph{type} of the expected
inputs. %, e.g., an integer, an string, or an abstract data type (ADT).
%
QuickCheck then generates a large number of random \emph{test cases} and reports
those violating testing properties.
%

% Situation with Quick Check, ADTs, tools
While QuickCheck comes equipped with random generators for built-in types, there
exists additional tools to automatically derive generators for user-defined
abstract data types (ADT) (e.g.,
\cite{mitchell2007,RuncimanNL08,DuregardJW12,grieco2017,DBLP:conf/haskell/MistaRH18}).
%
The main difference about these tools lies on the guarantees provided when it
comes to ensuring a terminating generation process or distribution of the random
values.
%
When generating ADT values, these tools guarantee that such values are
\emph{well-typed}.
%
In that manner, generated values follow the structure described by the
definition of the ADT, which proves useful when testing software with highly
structured inputs like compilers \cite{Palka11,MidtgaardJKNN17}.
%
Generating values from ADTs also proves to be a fruitful approach when looking
for vulnerabilities with fuzzers \cite{GriecoCB16,grieco2017}.
%

%% Problem


%% Our proposal

We remark that, altough this work focuses on Haskell data types, this technique
is general enough to be applied to most programming languages with some level of
support for composite types.


The main contribuitions of this paper are:
%
\begin{itemize}
  %
\item We identify two patological scenarios for which standard type-driven
  automatic derivation tools fail to synthesize practical random generators, due
  to a lack of either type structure or domain knowleadge (Section
  \ref{sec:randomtesting}).
  %
\item We present a generation technique able to encode stronger properties of
  the target data type by reifying the static information present on the program
  codebase (Section \ref{sec:hrep}).
  %
\item We apply and extend the theory of branching processes to analitically
  predict the average distribution of generated values.
  %
  Furthermore, we use the predictions to perform simulation-based optimization
  of the random generation parameters (Section \ref{sec:synthesis}).
  %
\item We provide an implementation of our ideas in the form of a Haskell library
  to perform automatic derivation of random generators capable to extract useful
  structure information from the user source code.
  %
\end{itemize}

% Local Variables:
% TeX-master: "main.lhs.tex"
% TeX-command-default: "Make"
% End:
