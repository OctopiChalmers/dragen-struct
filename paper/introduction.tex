\section{Introduction}

% Scenario of random testing
Random testing is a promising approach to finding bugs
\cite{HughesNSA16,HughesPAN16,ArtsHNS15}.
%
\quickcheck \cite{ClaessenH00} is the dominant tool of this sort used by the
Haskell community.
%
It requires developers to specify (i) \emph{testing properties} which describe
programs behavior and (ii) \emph{random data generator} based on the
\underline{\emph{types}} of the expected inputs (e.g., an integer, an string,
etc.). %, e.g., an integer, an string, or an abstract data type (ADT).
%
QuickCheck then generates random test cases and reports those violating testing
properties.
%

% Situation with Quick Check, ADTs, tools
QuickCheck comes equipped with random generators for built-in types, while it
requires to manually write generators for user-defined abstract data types
(ADTs).
%
Recently, there have been a proliferation of tools to automatically derive
QuickCheck generators for ADTs
\cite{mitchell2007,RuncimanNL08,DuregardJW12,grieco2017,DBLP:conf/haskell/MistaRH18}.
%
The main difference about these tools lies on the guarantees provided to ensure
\emph{the termination of the generation process} and the \emph{distribution of
  random values}.
%
Despite their differences, these tools guarantee that generated values are
\emph{well-typed}.
%
In other words, generated values follow the structure described by the
definition of the ADT.
%

%% How to use random generated ADTs
Well-typed ADT values are specially useful when testing programs which expect
highly structured inputs, e.g., compilers \cite{Palka11,MidtgaardJKNN17}.
%
Generating ADT values also proves fruitful when looking for vulnerabilities with
fuzzers \cite{GriecoCB16,grieco2017}.
%
%% Problem
%
Despite these success stories, ADT type-definitions do not often capture all the
invariants expected from the data that they are intended to model.
%
As a result, even if random values are well-typed, they might not present enough
structure to penetrate into deep layers of software.

%% Our proposal
In this work, we propose a novel improvement in the generation process of ADT
values by exploiting some static information found in the codebase.
%
More specifically, to refine the structure of generated values, we propose a
generation process that is capable to consider how programs pattern-matched on
ADTs values as well as how they get manipulated via interfaces.
%
Furthermore, we show how to predict (at compile time) the distribution of the
\emph{expected} numbers of ADT constructors, values fitting a certain pattern,
and calls to interfaces.
%
\todo[inline,author=Ale]{Mention branching processes}

%
We implement our ideas in a tool, called {\dragenp}, that is capable to
automatically synthesize QuickCheck generators for ADT values, where the
distributions of random values can be adjusted at compile-time to what
developers might want based on our predictions.
%
\todo[inline,author=Ale]{Add here later about test cases when we know what they
  are}

We remark that, although this work focuses on Haskell algebraic data types, this
technique is general enough to be applied to most programming languages.
%with some level of support for composite types.

% The main contribuitions of this paper are:
% %
% \begin{itemize}
%   %
% \item We identify two patological scenarios for which standard type-driven
%   automatic derivation tools fail to synthesize practical random generators, due
%   to a lack of either type structure or domain knowleadge (Section
%   \ref{sec:randomtesting}).
%   %
% \item We present a generation technique able to encode stronger properties of
%   the target data type by reifying the static information present on the program
%   codebase (Section \ref{sec:hrep}).
%   %
% \item We apply and extend the theory of branching processes to analitically
%   predict the average distribution of generated values.
%   %
%   Furthermore, we use the predictions to perform simulation-based optimization
%   of the random generation parameters (Section \ref{sec:synthesis}).
%   %
% \item We provide an implementation of our ideas in the form of a Haskell library
%   to perform automatic derivation of random generators capable to extract useful
%   structure information from the user source code.
%   %
% \end{itemize}

% Local Variables:
% TeX-master: "main.lhs.tex"
% TeX-command-default: "Make"
% End:
