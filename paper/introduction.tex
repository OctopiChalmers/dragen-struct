\section{Introduction}

Aliquam erat volutpat. Nunc eleifend leo vitae magna. In id erat non orci
commodo lobortis. Proin neque massa, cursus ut, gravida ut, lobortis eget,
lacus. Sed diam. Praesent fermentum tempor tellus. Nullam tempus. Mauris ac
felis vel velit tristique imperdiet. Donec at pede. Etiam vel neque nec dui
dignissim bibendum. Vivamus id enim. Phasellus neque orci, porta a, aliquet
quis, semper a, massa. Phasellus purus. Pellentesque tristique imperdiet tortor.
Nam euismod tellus id erat\cite{grieco2017}.


We remark that, altough this work focuses on Haskell data types, this technique
is general enough to be applied to most programming languages with some level of
support for composite types.


The main contribuitions of this paper are:
%
\begin{itemize}
  %
\item We identify two patological scenarios for which standard type-driven
  automatic derivation tools fail to synthesize practical random generators, due
  to a lack of either type structure or domain knowleadge (Section
  \ref{sec:randomtesting}).
  %
\item We present a generation technique able to encode stronger properties of
  the target data type by reifying the static information present on the program
  codebase (Section \ref{sec:hrep}).
  %
\item We apply and extend the theory of branching processes to analitically
  predict the average distribution of generated values.
  %
  Furthermore, we use the predictions to perform simulation-based optimization
  of the random generation parameters (Section \ref{sec:synthesis}).
  %
\item We provide an implementation of our ideas in the form of a Haskell library
  to perform automatic derivation of random generators capable to extract useful
  structure information from the user source code.
  %
\end{itemize}