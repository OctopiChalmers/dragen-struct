\section{Introduction}

Pellentesque dapibus suscipit ligula. Donec posuere augue in quam. Etiam vel
tortor sodales tellus ultricies commodo. Suspendisse potenti. Aenean in sem ac
leo mollis blandit. Donec neque quam, dignissim in, mollis nec, sagittis eu,
wisi. Phasellus lacus. Etiam laoreet quam sed arcu. Phasellus at dui in ligula
mollis ultricies. Integer placerat tristique nisl. Praesent augue. Fusce
commodo. Vestibulum convallis, lorem a tempus semper, dui dui euismod elit,
vitae placerat urna tortor vitae lacus. Nullam libero mauris, consequat quis,
varius et, dictum id, arcu. Mauris mollis tincidunt felis. Aliquam feugiat
tellus ut neque. Nulla facilisis, risus a rhoncus fermentum, tellus tellus
lacinia purus, et dictum nunc justo sit amet elit.

Lorem ipsum dolor sit amet, consectetuer adipiscing elit. Donec hendrerit tempor
tellus. Donec pretium posuere tellus. Proin quam nisl, tincidunt et, mattis
eget, convallis nec, purus. Cum sociis natoque penatibus et magnis dis
parturient montes, nascetur ridiculus mus. Nulla posuere. Donec vitae dolor.
Nullam tristique diam non turpis. Cras placerat accumsan nulla. Nullam rutrum.
Nam vestibulum accumsan nisl.

Aliquam erat volutpat. Nunc eleifend leo vitae magna. In id erat non orci
commodo lobortis. Proin neque massa, cursus ut, gravida ut, lobortis eget,
lacus. Sed diam. Praesent fermentum tempor tellus. Nullam tempus. Mauris ac
felis vel velit tristique imperdiet. Donec at pede. Etiam vel neque nec dui
dignissim bibendum. Vivamus id enim. Phasellus neque orci, porta a, aliquet
quis, semper a, massa. Phasellus purus. Pellentesque tristique imperdiet tortor.
Nam euismod tellus id erat\cite{grieco2017}.

The main contribuitions of this paper are:
%
\begin{itemize}
  %
\item We identify two patological scenarios for which standard type-driven
  automatic derivation tools fail to synthesize practical random generators, due
  to a lack of either type structure or domain knowleadge (Section
  \ref{sec:randomtesting}).
  %
\item We present a generation technique able to encode stronger properties of
  the target data by reifying the static information present on the program
  codebase (Section \ref{sec:hrep}).
  %
\item We apply and extend the theory of branching processes to analitically
  predict the average distribution of generated values.
  %
  Furthermore, we use the predictions to perform simulation-based optimization
  of the random generation parameters (Section \ref{sec:synthesis}).
  %
\item We provide an implementation of our ideas in the form of a Haskell library
  to perform automatic derivation of random generators capable to extract
  useful structure information from the user source code.
  %
\end{itemize}