\section{Extracting Structure} \label{sec:hrep}

In this section we present a compositional representation to express the random
generation of values following the internal structure of their data types along
with the structure present on patterns matchings and abstract interfaces.


The key idea of this work is to represent different sources of structured value
information of a data type in a homogeneous way that we call a ``higher-level
representation'' (|HRep| from now on).%
\footnote{The notion of higher-level comes from that the generation process of
  given target data type is entirely determined by the type of the chosen
  representation, instead of by a concrete generator defined at the term level.}
%
For this purpose, we use a series of automatically derived data types, each one
representing an atomic unit of information that can be randomly generated and
then reflected back to the corresponding value of the original data type.


We wil reuse the previously defined data type |Exp| and the function |foo| to
explain the different concepts involved all across this section.


\subsection{\textbf{Representing Data Constructors}}

We begin by introducing the simplest representation that we can extract from our
codebase: the representation of single data constructor.
%
Each original constructor can be represented by a automatically derived data
type consisting of a single constructor with the same fields, except for the
recursive ones that are abstracted away.
%
In this light, we represent each constructor of the data type |Exp| as follows:

\begin{code}
data HRep_Val  r = Mk_Val Int
data HRep_Add  r = Mk_Add r r
data HRep_Mul  r = Mk_Mul r r
\end{code}

Note that the previous definition are type parametric over the type parameter
|r|.
%
This way we can replace |r| with any concrete data type, obtaining different
possible values on each case.
%
For instance, the value (|Mk_Add 10 20|) has type |HRep_Add Int|, while the
value (|Mk_Mul True False|) has type |HRep_Mul Bool|.
%
In practice, this parametricity let us instatiate |a| with the type of the
chosen representation (which might change according to the user demands),
without having to modify anything in the underlying machinery.


For this purpose, we employ Haskell's type classes mechanism \tocite{}.
%
We define an evaluation type class $(\Downarrow)$ which specifies how to perform
a single transformation |step| from an |HRep| back into a concrete value of the
target:
%
\begin{align*}
  \llbracket \_ \rrbracket_{target}\ :\ rep_{target}\ \rightarrow\ target
\end{align*}

% \begin{code}
% class (rep down target) where
%   step :: rep target -> target
% \end{code}

Then, we can define the following overloaded instances of the |step| operation
for the canonical |HRep| of data constructors simply by mapping each lifted
constructor back into its corresponding one.
%
\begin{alignat*}{4}
  &\evalrep{|Mk_Val|\ n\ \ \ &&}{|Exp|}
    &&= |Val|\ n \\
  &\evalrep{|Mk_Add|\ x\ y   &&}{|Exp|}
    &&= |Add|\ \evalrep{x}{|Exp|}\ \evalrep{y}{|Exp|} \\
  &\evalrep{|Mk_Mul|\ x\ y   &&}{|Exp|}
    &&= |Mul|\ \evalrep{x}{|Exp|}\ \evalrep{y}{|Exp|}
\end{alignat*}


% \begin{code}
% instance (HRep_Val down Exp) where
%   step (Mk_Val n) = Val n

% instance (HRep_Add down Exp) where
%   step (Mk_Add x y) = Add x y

% instance (HRep_Mul down Exp) where
%   step (Mk_Mul x y) = Mul x y
% \end{code}

\begin{code}
  gen_Val  gen_r  = Mk_Val  <$> arbitrary
  gen_Add  gen_r  = Mk_Add  <$> smaller gen_r <*> smaller gen_r
  gen_Mul  gen_r  = Mk_Mul  <$> smaller gen_r <*> smaller gen_r
\end{code} %$

\subsection{\textbf{Type Level Combinators}}

In order to compose the |HRep|s for the single data constructors already
presented, we will need to define a series of type level combinators.

In first place, we define a type combinator ($\otimes$) to tag a |HRep| with an
explicit generation frequency:

\begin{code}
data (f otimes n) a = Freq (f a)
\end{code}

This combinator is evaluated back to our target data type simply by piping the
result from the inner representation.
%
It does not change the evaluation semantics, as it is only considered at
generation time:
%
\begin{align*}
  \evalrep{Freq\ x : f \otimes n}{t} = \evalrep{x : f}{t}
\end{align*}

\begin{code}
gen_f_times_n gen_r = Term <$> gen_f gen_r
\end{code}

% \begin{code}
% instance (f otimes n down Exp) where
%   step (Freq f) = step f
% \end{code}

% \begin{code}
% instance Arbitrary1 (f otimes n) where
%   liftArbitrary gen_HRep = Tag <$> gen_HRep
% \end{code} %$

When generating random values, we will only allow certain terminal
representations to be generated at the moment that the generation size gets
exhausted.
%
For this, we follow the same approach as before to define a combinator (|term|)
to tag terminal representations.

\begin{code}
data (f_term) a = Term (f a)
\end{code}

\begin{align*}
  \evalrep{Term\ x : |f_term|}{t} = \evalrep{x : f}{t}
\end{align*}

\begin{code}
gen_f_term gen_r = Term <$> gen_f gen_r
\end{code}


With these individual representations, we can define a type combinator
$(\oplus)$ to compose two |HRep|s into a single one:

\begin{code}
data (f oplus g) a = L (f a) | R (g a)
\end{code}

Then, a composite representation can be transformed back into the concrete
target type by pattern matching on the data type variant and applying the |step|
tranformation to the inner |HRep|:
%
\begin{alignat*}{3}
  &\evalrep{L\ x &&: f \oplus g}{t} = \evalrep{x : f &&}{t} \\
  &\evalrep{R\ x &&: f \oplus g}{t} = \evalrep{y : g &&}{t}
\end{alignat*}

\begin{code}
  gen_f_plus_g gen_r  = sized (\size ->
    if size == 0
    then frequency
      [ (freq0  at_f,  L  <$> gen_f  gen_r  )
      , (freq0  at_g,  R  <$> gen_g  gen_r  ) ]
    else frequency
      [ (freq   at_f,  L  <$> gen_f  gen_r  )
      , (freq   at_g,  R  <$> gen_g  gen_r  ) ])
\end{code} %$

% \begin{code}
% instance (f oplus g down Exp) where
%   step (L f) = step f
%   step (R g) = step g
% \end{code}

% \begin{code}
% instance Arbitrary1 (f oplus g) where
%   liftArbitrary gen_HRep
%     = frequency
%       [ (freq at_f, L <$> gen_HRep)
%       , (freq at_g, R <$> gen_HRep) ]
% \end{code} %$


With the introduced combinators, we can easily create a type synonym |HRep_Exp|
to refer to the canonical |HRep| of our original data type |Exp|, tagging for
instance the representation of the constuctor |Add| to be generated in with
double the frequency of rest:

\begin{code}
type HRep_Exp  =       HRep_Val_term
               oplus'  HRep_Add  otimes 2
               oplus'  HRep_Mul
\end{code}


\begin{code}
  data Fix f = Fix (f (Fix f))
\end{code}

\begin{align*}
  eval : Fix\ rep \rightarrow target
\end{align*}


\subsection{\textbf{Representing Pattern Matchings}}

\begin{code}
data HRep_foo_1  r = Mk_foo_1 r r
data HRep_foo_2  r = Mk_foo_2 Int r
\end{code}

\begin{align*}
  &\evalrep{|Mk_foo_1|\ x\ y}{Exp} = \\
  &\ \ |Add (Add eval_x_Exp (Val 50)) (Add (Val 25) eval_y_Exp)|\\
  &\evalrep{|Mk_foo_2|\ x\ y}{Exp} = \\
  &\ \ |Mul (Val 50) (Mul (Val x) eval_y_Exp)|
\end{align*}

% \begin{code}
% instance (HRep_foo_1 down Exp) where
%   step (Mk_foo_1 x y)
%     = Add (Add x (Val 50)) (Add (Val 25) y)

% instance (HRep_foo_2 down Exp) where
%   step (Mk_foo_2 x y)
%     = Mul (Val 50) (Mul (Val x) y)
% \end{code}

% \begin{code}
% instance Arbitrary1 HRep_foo_1 where
%   liftArbitrary gen_HRep
%     = Mk_foo_1 <$> gen_HRep <*> gen_HRep

% instance Arbitrary1 HRep_foo_2 where
%   liftArbitrary gen_HRep
%     = Mk_foo_2 <$> arbitrary <*> gen_HRep
% \end{code} %$


\begin{code}
type HRep_foo  = HRep_foo_1 oplus''  HRep_foo_2
\end{code}

\subsection{\textbf{Representing Abstract Interfaces}}

\begin{code}
module M where

ten :: Exp
ten = Val 10

square :: Exp -> Exp
square x = Mul x x

minus :: Exp -> Exp -> Exp
minus x y = Add x (Mul y (Val (-1)))
\end{code}

\begin{code}
data HRep_ten       r = Mk_ten
data HRep_square    r = Mk_square   r
data HRep_minus     r = Mk_minus    r r
\end{code}

\begin{alignat*}{3}
  &\evalrep{|Mk_ten|         &&}{Exp} = |ten| \\
  &\evalrep{|Mk_square|\ x   &&}{Exp} = |square eval_x_Exp| \\
  &\evalrep{|Mk_minus|\ x\ y &&}{Exp} = |minus  eval_x_Exp eval_y_Exp|
\end{alignat*}

% \begin{code}
% instance (HRep_ten down Exp) where
%   step Mk_ten = ten

% instance (HRep_square down Exp) where
%   step (Mk_square x) = square x

% instance (HRep_minus down Exp) where
%   step (Mk_minus x y) = minus x y
% \end{code}

% \begin{code}
% instance Arbitrary1 HRep_ten where
%   liftArbitrary gen_HRep = pure Mk_ten

% instance Arbitrary1 HRep_square where
%   liftArbitrary gen_HRep = Mk_square <$> gen_HRep

% instance Arbitrary1 HRep_minus where
%   liftArbitrary gen_HRep
%     = Mk_minus <$> gen_HRep <*> gen_HRep
% \end{code} %$

\begin{code}
type HRep_M  = HRep_ten oplus''  HRep_square oplus''  HRep_minus
\end{code}



\begin{code}
type Spec  =       HRep_Exp  otimes 4
           oplus'  HRep_foo  otimes 2
           oplus'  HRep_M
\end{code}

This previous definition can be interpreted graphically as it is shown in the
Figure \ref{fig:hrep}.
%
Curly arrows represent the structural information extracted using
meta-programming.

\begin{figure}[t]
  \centering
  \section{Extracting Structure} \label{sec:hrep}

In this work, we ...

\begin{code}
data HRep_Val  a = Mk_Val Int
data HRep_Add  a = Mk_Add a a
data HRep_Mul  a = Mk_Mul a a
\end{code}

For this purpose, we employ Haskell's type classes mechanism \tocite{}.
%
We define an evaluation type class $(\Downarrow)$ which specifies how to perform
a single transformation |step| from the higher-order representation back into a
concrete value of the target:

\begin{code}
class (rep down target) where
  step :: rep target -> target
\end{code}

Then, we can define the following overloaded instances of the |step| operation
for the canonical representations of data constructors simply by mapping each
lifted constructor back into its corresponding one.

\begin{code}
instance (HRep_Val down Exp) where
  step (Mk_Val n) = Val n

instance (HRep_Add down Exp) where
  step (Mk_Add x y) = Add x y

instance (HRep_Mul down Exp) where
  step (Mk_Mul x y) = Mul x y
\end{code}

With these individual representations, we can define a type combinator
$(\oplus)$ to compose two representations into a single one:

\begin{code}
data (f oplus g) a = L (f a) | R (g a)
\end{code}

Then, a composite representation can be transformed back into the concrete
target type by pattern matching on the data type variant and applying the |step|
tranformation to the inner representation:

\begin{code}
instance (f oplus g down Exp) where
  step (L f) = step f
  step (R g) = step g
\end{code}

Furthermore, we can define a type combinator ($\otimes$) to tag every
representation with an explicit generation frequency:

\begin{code}
data (f otimes n) a = Freq (f a)
\end{code}

This combinator is evaluated back to our target data type simply by piping the
result from the inner representation.
%
It does not change the evaluation semantics, as it is only considered at
generation time:

\begin{code}
instance (f otimes n down Exp) where
  step (Freq f) = step f
\end{code}

With the introduced combinators, we can easily create a type synonym |HRep_Exp|
to refer to the canonical representation of our original data type |Exp|,
tagging for instance the representation of the constuctor |Add| to be generated
in double the proportion of rest of the representations:

\begin{code}
type HRep_Exp  =       HRep_Val
               oplus'  HRep_Add  otimes 2
               oplus'  HRep_Mul
\end{code}



\begin{code}
data HRep_foo_1  a = Mk_foo_1 a a
data HRep_foo_2  a = Mk_foo_2 Int a
\end{code}

\begin{code}
instance (HRep_foo_1 down Exp) where
  step (Mk_foo_1 x y)
    = Add (Add x (Val 50)) (Add (Val 25) y)

instance (HRep_foo_2 down Exp) where
  step (Mk_foo_2 x y)
    = Mul (Val 50) (Mul (Val x) y)
\end{code}

\begin{code}
type HRep_foo  =       HRep_foo_1
               oplus'  HRep_foo_2
\end{code}


\begin{code}
data HRep_ten       a = Mk_ten
data HRep_square    a = Mk_square   a
data HRep_minus     a = Mk_minus    a a
\end{code}

\begin{code}
instance (HRep_ten down Exp) where
  step Mk_ten = ten

instance (HRep_square down Exp) where
  step (Mk_square x) = square x

instance (HRep_minus down Exp) where
  step (Mk_minus x y) = minus x y
\end{code}


\begin{code}
type HRep_M  =       HRep_ten
             oplus'  HRep_square
             oplus'  HRep_minus
\end{code}



\begin{code}
type Spec_Exp  =       HRep_Exp  otimes 4
               oplus'  HRep_foo  otimes 2
               oplus'  HRep_M
\end{code}

This previous definition can be interpreted graphically as it is shown in the
Figure \ref{fig:hrep}.
%
Curly arrows represent the structural information extracted using
meta-programming.

\begin{figure}[t]
  \centering
  \section{Extracting Structure} \label{sec:hrep}

In this work, we ...

\begin{code}
data HRep_Val  a = Mk_Val Int
data HRep_Add  a = Mk_Add a a
data HRep_Mul  a = Mk_Mul a a
\end{code}

For this purpose, we employ Haskell's type classes mechanism \tocite{}.
%
We define an evaluation type class $(\Downarrow)$ which specifies how to perform
a single transformation |step| from the higher-order representation back into a
concrete value of the target:

\begin{code}
class (rep down target) where
  step :: rep target -> target
\end{code}

Then, we can define the following overloaded instances of the |step| operation
for the canonical representations of data constructors simply by mapping each
lifted constructor back into its corresponding one.

\begin{code}
instance (HRep_Val down Exp) where
  step (Mk_Val n) = Val n

instance (HRep_Add down Exp) where
  step (Mk_Add x y) = Add x y

instance (HRep_Mul down Exp) where
  step (Mk_Mul x y) = Mul x y
\end{code}

With these individual representations, we can define a type combinator
$(\oplus)$ to compose two representations into a single one:

\begin{code}
data (f oplus g) a = L (f a) | R (g a)
\end{code}

Then, a composite representation can be transformed back into the concrete
target type by pattern matching on the data type variant and applying the |step|
tranformation to the inner representation:

\begin{code}
instance (f oplus g down Exp) where
  step (L f) = step f
  step (R g) = step g
\end{code}

Furthermore, we can define a type combinator ($\otimes$) to tag every
representation with an explicit generation frequency:

\begin{code}
data (f otimes n) a = Freq (f a)
\end{code}

This combinator is evaluated back to our target data type simply by piping the
result from the inner representation.
%
It does not change the evaluation semantics, as it is only considered at
generation time:

\begin{code}
instance (f otimes n down Exp) where
  step (Freq f) = step f
\end{code}

With the introduced combinators, we can easily create a type synonym |HRep_Exp|
to refer to the canonical representation of our original data type |Exp|,
tagging for instance the representation of the constuctor |Add| to be generated
in double the proportion of rest of the representations:

\begin{code}
type HRep_Exp  =       HRep_Val
               oplus'  HRep_Add  otimes 2
               oplus'  HRep_Mul
\end{code}



\begin{code}
data HRep_foo_1  a = Mk_foo_1 a a
data HRep_foo_2  a = Mk_foo_2 Int a
\end{code}

\begin{code}
instance (HRep_foo_1 down Exp) where
  step (Mk_foo_1 x y)
    = Add (Add x (Val 50)) (Add (Val 25) y)

instance (HRep_foo_2 down Exp) where
  step (Mk_foo_2 x y)
    = Mul (Val 50) (Mul (Val x) y)
\end{code}

\begin{code}
type HRep_foo  =       HRep_foo_1
               oplus'  HRep_foo_2
\end{code}


\begin{code}
data HRep_ten       a = Mk_ten
data HRep_square    a = Mk_square   a
data HRep_minus     a = Mk_minus    a a
\end{code}

\begin{code}
instance (HRep_ten down Exp) where
  step Mk_ten = ten

instance (HRep_square down Exp) where
  step (Mk_square x) = square x

instance (HRep_minus down Exp) where
  step (Mk_minus x y) = minus x y
\end{code}


\begin{code}
type HRep_M  =       HRep_ten
             oplus'  HRep_square
             oplus'  HRep_minus
\end{code}



\begin{code}
type Spec_Exp  =       HRep_Exp  otimes 4
               oplus'  HRep_foo  otimes 2
               oplus'  HRep_M
\end{code}

This previous definition can be interpreted graphically as it is shown in the
Figure \ref{fig:hrep}.
%
Curly arrows represent the structural information extracted using
meta-programming.

\begin{figure}[t]
  \centering
  \section{Extracting Structure} \label{sec:hrep}

In this work, we ...

\begin{code}
data HRep_Val  a = Mk_Val Int
data HRep_Add  a = Mk_Add a a
data HRep_Mul  a = Mk_Mul a a
\end{code}

For this purpose, we employ Haskell's type classes mechanism \tocite{}.
%
We define an evaluation type class $(\Downarrow)$ which specifies how to perform
a single transformation |step| from the higher-order representation back into a
concrete value of the target:

\begin{code}
class (rep down target) where
  step :: rep target -> target
\end{code}

Then, we can define the following overloaded instances of the |step| operation
for the canonical representations of data constructors simply by mapping each
lifted constructor back into its corresponding one.

\begin{code}
instance (HRep_Val down Exp) where
  step (Mk_Val n) = Val n

instance (HRep_Add down Exp) where
  step (Mk_Add x y) = Add x y

instance (HRep_Mul down Exp) where
  step (Mk_Mul x y) = Mul x y
\end{code}

With these individual representations, we can define a type combinator
$(\oplus)$ to compose two representations into a single one:

\begin{code}
data (f oplus g) a = L (f a) | R (g a)
\end{code}

Then, a composite representation can be transformed back into the concrete
target type by pattern matching on the data type variant and applying the |step|
tranformation to the inner representation:

\begin{code}
instance (f oplus g down Exp) where
  step (L f) = step f
  step (R g) = step g
\end{code}

Furthermore, we can define a type combinator ($\otimes$) to tag every
representation with an explicit generation frequency:

\begin{code}
data (f otimes n) a = Freq (f a)
\end{code}

This combinator is evaluated back to our target data type simply by piping the
result from the inner representation.
%
It does not change the evaluation semantics, as it is only considered at
generation time:

\begin{code}
instance (f otimes n down Exp) where
  step (Freq f) = step f
\end{code}

With the introduced combinators, we can easily create a type synonym |HRep_Exp|
to refer to the canonical representation of our original data type |Exp|,
tagging for instance the representation of the constuctor |Add| to be generated
in double the proportion of rest of the representations:

\begin{code}
type HRep_Exp  =       HRep_Val
               oplus'  HRep_Add  otimes 2
               oplus'  HRep_Mul
\end{code}



\begin{code}
data HRep_foo_1  a = Mk_foo_1 a a
data HRep_foo_2  a = Mk_foo_2 Int a
\end{code}

\begin{code}
instance (HRep_foo_1 down Exp) where
  step (Mk_foo_1 x y)
    = Add (Add x (Val 50)) (Add (Val 25) y)

instance (HRep_foo_2 down Exp) where
  step (Mk_foo_2 x y)
    = Mul (Val 50) (Mul (Val x) y)
\end{code}

\begin{code}
type HRep_foo  =       HRep_foo_1
               oplus'  HRep_foo_2
\end{code}


\begin{code}
data HRep_ten       a = Mk_ten
data HRep_square    a = Mk_square   a
data HRep_minus     a = Mk_minus    a a
\end{code}

\begin{code}
instance (HRep_ten down Exp) where
  step Mk_ten = ten

instance (HRep_square down Exp) where
  step (Mk_square x) = square x

instance (HRep_minus down Exp) where
  step (Mk_minus x y) = minus x y
\end{code}


\begin{code}
type HRep_M  =       HRep_ten
             oplus'  HRep_square
             oplus'  HRep_minus
\end{code}



\begin{code}
type Spec_Exp  =       HRep_Exp  otimes 4
               oplus'  HRep_foo  otimes 2
               oplus'  HRep_M
\end{code}

This previous definition can be interpreted graphically as it is shown in the
Figure \ref{fig:hrep}.
%
Curly arrows represent the structural information extracted using
meta-programming.

\begin{figure}[t]
  \centering
  \input{tikz/hrep.tex}
  \caption{Higher order representation of the data type |Exp|, using structural
    information from the function |foo| and the abstract interface of the module
    |M|.}
  \label{fig:hrep}
\end{figure}

% \begin{code}
% data (Term f) a = TagTerm (f a)

% instance (Term f down Exp) where
%   step (TagTerm f) = step f
% \end{code}
  \caption{Higher order representation of the data type |Exp|, using structural
    information from the function |foo| and the abstract interface of the module
    |M|.}
  \label{fig:hrep}
\end{figure}

% \begin{code}
% data (Term f) a = TagTerm (f a)

% instance (Term f down Exp) where
%   step (TagTerm f) = step f
% \end{code}
  \caption{Higher order representation of the data type |Exp|, using structural
    information from the function |foo| and the abstract interface of the module
    |M|.}
  \label{fig:hrep}
\end{figure}

% \begin{code}
% data (Term f) a = TagTerm (f a)

% instance (Term f down Exp) where
%   step (TagTerm f) = step f
% \end{code}
  \caption{Higher level representation of the data type |Exp|, defined using
    structural information from the function |foo| and the abstract interface of
    the module |M|.}
  \label{fig:hrep}
\end{figure}

% \begin{code}
% data (Term f) a = TagTerm (f a)

% instance (Term f down Exp) where
%   step (TagTerm f) = step f
% \end{code}