\section{Related Work}

%The literature on random generation of structured data is vast, both in
%practical and in theoretical terms.

%
% Boltzmann models
%
Boltzmann models \cite{Duchon2004} are a general approach to randomly generating
combinatorial structures such as trees and graphs, closed simply-typed lambda
terms, etc.
%
A random generator built around such models uniformly generates values of a
target size with a certain size tolerance.
%
However, it has been argued that this approach has theoretical and practical
limitations in the context of software testing \cite{feldt2013}.
%
In a recent work, Bendkowski et al. provides a framework called
\emph{boltzmann-brain} to specify and synthesize standalone Haskell random
generators based on Boltzmann models \cite{bendkowski2018}.
%
This framework mixes parameter tuning and rejection of samples of unwanted sizes
to approximate the desired distribution of values according to user demands.
%
The overall discard ratio then depends on how constrained the desired sizes of
values are.
%
On the other hand, our work is focused on approximating the desired distribution
as much as possible via parameter optimization, without discarding any generated
value at runtime.
%
Although promising, we found difficulties to compare both approaches in practice
due that \emph{boltzmann-brain} is considered a conceptual standalone utility
that produces self-contained samplers.
%
In this light, data specifications have to be manually written using a special
syntax, and cannot include Haskell ground types like |String| or |Int|,
difficulting the integration of this tool to existing Haskell codebases like the
ones we consider in this work.



%
% GodelTest
%
From the practical point of view, Feldt and Poulding propose \emph{G\"odelTest}
\cite{feldt2013}, a search-based framework for generating biased data.
%
% \emph{G\"odelTest} uses non-determinism to generate a wide range of data structures.
%
Similar to our approach, \emph{G\"odelTest} works by optimizing the parameters
governing the desired biases on the generated data.
%
However, the optimization mechanism uses meta-heuristic search to find the best
parameters at runtime.
%
\dragenp on the other hand implements an analytic and composable prediction
mechanism that is only used at compile time to optimize the generation
parameters, thus avoiding performing any kind of runtime reinforcement.


%
% DART
%
Directed Automated Random Testing (DART) is a technique that combines random
testing with symbolic execution for C programs \cite{godefroid2005dart}.
%
It requires instrumenting the target programs in order to introduce testing
assertions and obtain feedback from previous testing executions, which is used
to explore new paths in the source code.
%
This technique has been shown to be remarkably useful, although it forces a
strong coupling between the testing suite and the target code.
%
Our tool intends to provide better random generation of values following an
undirected fashion, without having to instrument the target code, but still
extracting useful structural information from it.
