\section{Related Work}

The literature on random generation of structured data is vast, both in
practical and in theoretical terms.

%
% Boltzmann models
%
Boltzmann models \cite{Duchon2004} are a general approach to randomly generating
combinatorial structures such as trees and graphs, closed simply-typed lambda
terms, etc.
%
A random generator built around such models uniformly generates values of a
target size with a certain size tolerance.
%
However, it has been argued that this approach has theoretical and practical
limitations in the context of software testing \cite{feldt2013}.
%
In a recent work, Bendkowski provides a framework called \emph{boltzmann-brain}
to specify and synthesize standalone Haskell random generators based on
Boltzmann models \cite{bendkowski2018}.
%
This framework mixes parameter tunning and rejection of unwanted samples to
approximate the desired distribution of values according to user demands.
%
In this approach, the overall discard ratio depends on how constrained is this
desired distribution.
%
On the other hand, our work is focused on approximating the desired distribution
as much as possible via parameter optimization, without discarding any generated
value at runtime.
%
Altough promising, we found several difficulties to compare both approaches in
practice due that boltzmann-brain is considered a standalone utility that
produces self-contained samplers.
%
In this light, data specifications have to be manually written using a special
syntax, and cannot include Haskell ground types like |String| or |Int|,
difficulting the task of integrating this tool to existing Haskel codebases like
the ones we evaluate on this work.



%
% GodelTest
%
From the practical point of view, Feldt and Poulding propose G\"odelTest
\cite{feldt2013}, a search-based framework for generating biased data.
%
G\"odelTest uses non-determinism to generate a wide range of data structures.
%
Similar to our approach, it works optimizing the parameters governing the
desired biases on the generated data.
%
However, the optimization mechanism uses metaheuristic search for the best
parameters at runtime.
%
\dragenp on the other hand implements a analytic and composable prediction
mechanism that is used only used at compile time to optimize the generation
parameters, thus avoiding to perform any kind of runtime reinforcement.


%
% DART
%
Directed Automated Software Testing (DART) is a technique that combines random
testing with symbolic execution for C programs \cite{godefroid2005dart}.
%
I requires to instrument the target programs in order to introduce testing
assertions and obtain feedback from previous testing executions that is used to
explore new paths in the source code.
%
This technique has been shown to be remarkably useful, altough it forces a
strong coupling between the testing suite and the target code.
%
Our tool intends to provide better random generation of values following an
undirected fashion, without having to instrument the target code, but still
extracting useful structural information from it.
